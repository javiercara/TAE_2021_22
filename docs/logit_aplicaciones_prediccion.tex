% Options for packages loaded elsewhere
\PassOptionsToPackage{unicode}{hyperref}
\PassOptionsToPackage{hyphens}{url}
%
\documentclass[
]{article}
\title{Aplicaciones del modelo de regresión lineal: cálculo de
predicciones}
\author{}
\date{\vspace{-2.5em}}

\usepackage{amsmath,amssymb}
\usepackage{lmodern}
\usepackage{iftex}
\ifPDFTeX
  \usepackage[T1]{fontenc}
  \usepackage[utf8]{inputenc}
  \usepackage{textcomp} % provide euro and other symbols
\else % if luatex or xetex
  \usepackage{unicode-math}
  \defaultfontfeatures{Scale=MatchLowercase}
  \defaultfontfeatures[\rmfamily]{Ligatures=TeX,Scale=1}
\fi
% Use upquote if available, for straight quotes in verbatim environments
\IfFileExists{upquote.sty}{\usepackage{upquote}}{}
\IfFileExists{microtype.sty}{% use microtype if available
  \usepackage[]{microtype}
  \UseMicrotypeSet[protrusion]{basicmath} % disable protrusion for tt fonts
}{}
\makeatletter
\@ifundefined{KOMAClassName}{% if non-KOMA class
  \IfFileExists{parskip.sty}{%
    \usepackage{parskip}
  }{% else
    \setlength{\parindent}{0pt}
    \setlength{\parskip}{6pt plus 2pt minus 1pt}}
}{% if KOMA class
  \KOMAoptions{parskip=half}}
\makeatother
\usepackage{xcolor}
\IfFileExists{xurl.sty}{\usepackage{xurl}}{} % add URL line breaks if available
\IfFileExists{bookmark.sty}{\usepackage{bookmark}}{\usepackage{hyperref}}
\hypersetup{
  pdftitle={Aplicaciones del modelo de regresión lineal: cálculo de predicciones},
  hidelinks,
  pdfcreator={LaTeX via pandoc}}
\urlstyle{same} % disable monospaced font for URLs
\usepackage[margin=1in]{geometry}
\usepackage{color}
\usepackage{fancyvrb}
\newcommand{\VerbBar}{|}
\newcommand{\VERB}{\Verb[commandchars=\\\{\}]}
\DefineVerbatimEnvironment{Highlighting}{Verbatim}{commandchars=\\\{\}}
% Add ',fontsize=\small' for more characters per line
\usepackage{framed}
\definecolor{shadecolor}{RGB}{248,248,248}
\newenvironment{Shaded}{\begin{snugshade}}{\end{snugshade}}
\newcommand{\AlertTok}[1]{\textcolor[rgb]{0.94,0.16,0.16}{#1}}
\newcommand{\AnnotationTok}[1]{\textcolor[rgb]{0.56,0.35,0.01}{\textbf{\textit{#1}}}}
\newcommand{\AttributeTok}[1]{\textcolor[rgb]{0.77,0.63,0.00}{#1}}
\newcommand{\BaseNTok}[1]{\textcolor[rgb]{0.00,0.00,0.81}{#1}}
\newcommand{\BuiltInTok}[1]{#1}
\newcommand{\CharTok}[1]{\textcolor[rgb]{0.31,0.60,0.02}{#1}}
\newcommand{\CommentTok}[1]{\textcolor[rgb]{0.56,0.35,0.01}{\textit{#1}}}
\newcommand{\CommentVarTok}[1]{\textcolor[rgb]{0.56,0.35,0.01}{\textbf{\textit{#1}}}}
\newcommand{\ConstantTok}[1]{\textcolor[rgb]{0.00,0.00,0.00}{#1}}
\newcommand{\ControlFlowTok}[1]{\textcolor[rgb]{0.13,0.29,0.53}{\textbf{#1}}}
\newcommand{\DataTypeTok}[1]{\textcolor[rgb]{0.13,0.29,0.53}{#1}}
\newcommand{\DecValTok}[1]{\textcolor[rgb]{0.00,0.00,0.81}{#1}}
\newcommand{\DocumentationTok}[1]{\textcolor[rgb]{0.56,0.35,0.01}{\textbf{\textit{#1}}}}
\newcommand{\ErrorTok}[1]{\textcolor[rgb]{0.64,0.00,0.00}{\textbf{#1}}}
\newcommand{\ExtensionTok}[1]{#1}
\newcommand{\FloatTok}[1]{\textcolor[rgb]{0.00,0.00,0.81}{#1}}
\newcommand{\FunctionTok}[1]{\textcolor[rgb]{0.00,0.00,0.00}{#1}}
\newcommand{\ImportTok}[1]{#1}
\newcommand{\InformationTok}[1]{\textcolor[rgb]{0.56,0.35,0.01}{\textbf{\textit{#1}}}}
\newcommand{\KeywordTok}[1]{\textcolor[rgb]{0.13,0.29,0.53}{\textbf{#1}}}
\newcommand{\NormalTok}[1]{#1}
\newcommand{\OperatorTok}[1]{\textcolor[rgb]{0.81,0.36,0.00}{\textbf{#1}}}
\newcommand{\OtherTok}[1]{\textcolor[rgb]{0.56,0.35,0.01}{#1}}
\newcommand{\PreprocessorTok}[1]{\textcolor[rgb]{0.56,0.35,0.01}{\textit{#1}}}
\newcommand{\RegionMarkerTok}[1]{#1}
\newcommand{\SpecialCharTok}[1]{\textcolor[rgb]{0.00,0.00,0.00}{#1}}
\newcommand{\SpecialStringTok}[1]{\textcolor[rgb]{0.31,0.60,0.02}{#1}}
\newcommand{\StringTok}[1]{\textcolor[rgb]{0.31,0.60,0.02}{#1}}
\newcommand{\VariableTok}[1]{\textcolor[rgb]{0.00,0.00,0.00}{#1}}
\newcommand{\VerbatimStringTok}[1]{\textcolor[rgb]{0.31,0.60,0.02}{#1}}
\newcommand{\WarningTok}[1]{\textcolor[rgb]{0.56,0.35,0.01}{\textbf{\textit{#1}}}}
\usepackage{graphicx}
\makeatletter
\def\maxwidth{\ifdim\Gin@nat@width>\linewidth\linewidth\else\Gin@nat@width\fi}
\def\maxheight{\ifdim\Gin@nat@height>\textheight\textheight\else\Gin@nat@height\fi}
\makeatother
% Scale images if necessary, so that they will not overflow the page
% margins by default, and it is still possible to overwrite the defaults
% using explicit options in \includegraphics[width, height, ...]{}
\setkeys{Gin}{width=\maxwidth,height=\maxheight,keepaspectratio}
% Set default figure placement to htbp
\makeatletter
\def\fps@figure{htbp}
\makeatother
\setlength{\emergencystretch}{3em} % prevent overfull lines
\providecommand{\tightlist}{%
  \setlength{\itemsep}{0pt}\setlength{\parskip}{0pt}}
\setcounter{secnumdepth}{5}
\ifLuaTeX
  \usepackage{selnolig}  % disable illegal ligatures
\fi

\begin{document}
\maketitle

{
\setcounter{tocdepth}{2}
\tableofcontents
}
\hypertarget{predicciuxf3n-de-pi_i}{%
\section{\texorpdfstring{Predicción de
\(\pi_i\)}{Predicción de \textbackslash pi\_i}}\label{predicciuxf3n-de-pi_i}}

Sea el modelo de regresión logística

\[
P(Y_i = y_i) = \pi_i^{y_i} (1 - \pi_i)^{1-y_i}, \quad y_i = 0,1, \quad i = 1,2,\ldots,n
\]

donde:

\[
\pi_i = \frac{exp(x_i^T \beta)}{1 + exp(x_i^T \beta)}
\]

\[
x_i = 
\begin{bmatrix}
1 \\ x_{1i} \\ x_{2i} \\ \cdots \\ x_{ki}
\end{bmatrix}
, \quad
\beta = 
\begin{bmatrix}
\beta_0 \\ \beta_1 \\ \beta_2 \\ \cdots \\ \beta_k
\end{bmatrix}
\]

Estamos interesados en el valor de la respuesta para los regresores
\(x_p^T = [1 \ x_{1p} \ x_{2p} \ \cdots \ x_{kp}]\). Por tanto, el valor
predicho de \(\pi_i\) en \(x_p\) es:

\[
\hat \pi_p = \frac{exp(x_p^T \hat \beta)}{1 + exp(x_p^T \hat \beta)}
\]

donde \(\hat \beta\) es el vector de parámetros estimados:

\[
\hat \beta = 
\begin{bmatrix}
\hat \beta_0 \\ \hat \beta_1 \\ \hat \beta_2 \\ \cdots \\ \hat \beta_k
\end{bmatrix}
\]

\hypertarget{intervalo-de-confianza-para-pi_p}{%
\section{\texorpdfstring{Intervalo de confianza para
\(\pi_p\)}{Intervalo de confianza para \textbackslash pi\_p}}\label{intervalo-de-confianza-para-pi_p}}

Se tiene que

\[
\hat \beta \sim N(\beta,(X^T W X)^{-1})
\]

Por tanto

\[
x_p^T \hat \beta \sim N(x_p^T \beta, x_p^T (X^T W X)^{-1} x_p)
\] ya que

\[
E[x_p^T \hat \beta] = x_p^T E[\hat \beta] = x_p^T \beta
\] y

\[
Var[x_p^T \hat \beta]  = x_p^T Var[\hat \beta] x_p = x_p^T (X^T W X)^{-1} x_p
\]

Por tanto, el intervalo de confianza para \(x_p^T \beta\) es

\[
x_p^T \hat \beta - z_{\alpha/2} \sqrt{x_p^T (X^T W X)^{-1} x_p} \leq x_p^T \beta \leq x_p^T \hat \beta + z_{\alpha/2} \sqrt{x_p^T (X^T W X)^{-1} x_p}
\]

Si llamamos:

\[
L_p = x_p^T \hat \beta - z_{\alpha/2} \sqrt{x_p^T (X^T W X)^{-1} x_p} \\
U_p = x_p^T \hat \beta + z_{\alpha/2} \sqrt{x_p^T (X^T W X)^{-1} x_p}
\]

se tiene que

\[
\frac{exp(L_p)}{1+exp(L_p)} \leq \pi_p \leq \frac{exp(U_p)}{1+exp(U_p)}
\]

donde se recuerda que

\[
\pi_p = \frac{exp(x_p^T \beta)}{1 + exp(x_p^T \beta)}
\]

\hypertarget{ejemplos}{%
\section{Ejemplos}\label{ejemplos}}

\hypertarget{variable-respuesta-binaria}{%
\subsection{Variable respuesta
binaria}\label{variable-respuesta-binaria}}

\begin{Shaded}
\begin{Highlighting}[]
\NormalTok{d }\OtherTok{=} \FunctionTok{read.csv}\NormalTok{(}\StringTok{"datos/MichelinNY.csv"}\NormalTok{)}
\end{Highlighting}
\end{Shaded}

Primero estimamos el modelo:

\begin{Shaded}
\begin{Highlighting}[]
\NormalTok{m1 }\OtherTok{=} \FunctionTok{glm}\NormalTok{(InMichelin }\SpecialCharTok{\textasciitilde{}}\NormalTok{ Food }\SpecialCharTok{+}\NormalTok{ Decor }\SpecialCharTok{+}\NormalTok{ Service }\SpecialCharTok{+}\NormalTok{ Price, }\AttributeTok{data =}\NormalTok{ d, }\AttributeTok{family =}\NormalTok{ binomial)}
\FunctionTok{summary}\NormalTok{(m1)}
\end{Highlighting}
\end{Shaded}

\begin{verbatim}
## 
## Call:
## glm(formula = InMichelin ~ Food + Decor + Service + Price, family = binomial, 
##     data = d)
## 
## Deviance Residuals: 
##     Min       1Q   Median       3Q      Max  
## -3.3923  -0.6723  -0.3810   0.7169   1.9694  
## 
## Coefficients:
##              Estimate Std. Error z value Pr(>|z|)    
## (Intercept) -11.19745    2.30896  -4.850 1.24e-06 ***
## Food          0.40485    0.13146   3.080  0.00207 ** 
## Decor         0.09997    0.08919   1.121  0.26235    
## Service      -0.19242    0.12357  -1.557  0.11942    
## Price         0.09172    0.03175   2.889  0.00387 ** 
## ---
## Signif. codes:  0 '***' 0.001 '**' 0.01 '*' 0.05 '.' 0.1 ' ' 1
## 
## (Dispersion parameter for binomial family taken to be 1)
## 
##     Null deviance: 225.79  on 163  degrees of freedom
## Residual deviance: 148.40  on 159  degrees of freedom
## AIC: 158.4
## 
## Number of Fisher Scoring iterations: 6
\end{verbatim}

Queremos calcular la predicción en Food = 22, Decor = 25, Service = 24,
Price = 75:

\begin{Shaded}
\begin{Highlighting}[]
\NormalTok{xp }\OtherTok{=} \FunctionTok{c}\NormalTok{(}\DecValTok{1}\NormalTok{,}\DecValTok{22}\NormalTok{,}\DecValTok{25}\NormalTok{,}\DecValTok{24}\NormalTok{,}\DecValTok{75}\NormalTok{)}
\NormalTok{beta\_e }\OtherTok{=} \FunctionTok{coef}\NormalTok{(m1)}
\NormalTok{( }\AttributeTok{pi\_p =} \FunctionTok{exp}\NormalTok{(}\FunctionTok{t}\NormalTok{(xp) }\SpecialCharTok{\%*\%}\NormalTok{ beta\_e)}\SpecialCharTok{/}\NormalTok{(}\DecValTok{1} \SpecialCharTok{+} \FunctionTok{exp}\NormalTok{(}\FunctionTok{t}\NormalTok{(xp) }\SpecialCharTok{\%*\%}\NormalTok{ beta\_e)) ) }
\end{Highlighting}
\end{Shaded}

\begin{verbatim}
##           [,1]
## [1,] 0.9219606
\end{verbatim}

Para calcular el intervalo de confianza:

\begin{Shaded}
\begin{Highlighting}[]
\FunctionTok{source}\NormalTok{(}\StringTok{"logit\_funciones.R"}\NormalTok{)}
\NormalTok{H }\OtherTok{=} \FunctionTok{hess\_logL}\NormalTok{(}\FunctionTok{coef}\NormalTok{(m1),}\FunctionTok{model.matrix}\NormalTok{(m1))}
\NormalTok{xp }\OtherTok{=} \FunctionTok{matrix}\NormalTok{(xp, }\AttributeTok{ncol =} \DecValTok{1}\NormalTok{)}
\NormalTok{(}\AttributeTok{se =} \FunctionTok{sqrt}\NormalTok{(}\SpecialCharTok{{-}} \FunctionTok{t}\NormalTok{(xp) }\SpecialCharTok{\%*\%} \FunctionTok{solve}\NormalTok{(H) }\SpecialCharTok{\%*\%}\NormalTok{ xp ))}
\end{Highlighting}
\end{Shaded}

\begin{verbatim}
##           [,1]
## [1,] 0.6718705
\end{verbatim}

\begin{Shaded}
\begin{Highlighting}[]
\NormalTok{alfa }\OtherTok{=} \FloatTok{0.05}
\NormalTok{Lp }\OtherTok{=} \FunctionTok{t}\NormalTok{(xp) }\SpecialCharTok{\%*\%}\NormalTok{ beta\_e }\SpecialCharTok{{-}} \FunctionTok{qnorm}\NormalTok{(}\DecValTok{1}\SpecialCharTok{{-}}\NormalTok{alfa}\SpecialCharTok{/}\DecValTok{2}\NormalTok{)}\SpecialCharTok{*}\NormalTok{se}
\NormalTok{Up }\OtherTok{=} \FunctionTok{t}\NormalTok{(xp) }\SpecialCharTok{\%*\%}\NormalTok{ beta\_e }\SpecialCharTok{+} \FunctionTok{qnorm}\NormalTok{(}\DecValTok{1}\SpecialCharTok{{-}}\NormalTok{alfa}\SpecialCharTok{/}\DecValTok{2}\NormalTok{)}\SpecialCharTok{*}\NormalTok{se}
\CommentTok{\# limite inferior intrevalo confianza}
\FunctionTok{exp}\NormalTok{(Lp)}\SpecialCharTok{/}\NormalTok{(}\DecValTok{1}\SpecialCharTok{+}\FunctionTok{exp}\NormalTok{(Lp))}
\end{Highlighting}
\end{Shaded}

\begin{verbatim}
##           [,1]
## [1,] 0.7599576
\end{verbatim}

\begin{Shaded}
\begin{Highlighting}[]
\CommentTok{\# limite superior intrevalo confianza}
\FunctionTok{exp}\NormalTok{(Up)}\SpecialCharTok{/}\NormalTok{(}\DecValTok{1}\SpecialCharTok{+}\FunctionTok{exp}\NormalTok{(Up))}
\end{Highlighting}
\end{Shaded}

\begin{verbatim}
##           [,1]
## [1,] 0.9778199
\end{verbatim}

Las predicciones son:

\begin{Shaded}
\begin{Highlighting}[]
\NormalTok{xp }\OtherTok{=} \FunctionTok{data.frame}\NormalTok{(}\AttributeTok{Food =} \DecValTok{22}\NormalTok{, }\AttributeTok{Decor =} \DecValTok{25}\NormalTok{, }\AttributeTok{Service =} \DecValTok{24}\NormalTok{, }\AttributeTok{Price =} \DecValTok{75}\NormalTok{)}
\NormalTok{(}\AttributeTok{pred =} \FunctionTok{predict}\NormalTok{(m1, }\AttributeTok{newdata =}\NormalTok{ xp, }\AttributeTok{type =} \StringTok{"response"}\NormalTok{, }\AttributeTok{se.fit =}\NormalTok{ T))}
\end{Highlighting}
\end{Shaded}

\begin{verbatim}
## $fit
##         1 
## 0.9219606 
## 
## $se.fit
##          1 
## 0.04834054 
## 
## $residual.scale
## [1] 1
\end{verbatim}

\begin{Shaded}
\begin{Highlighting}[]
\NormalTok{pred}\SpecialCharTok{$}\NormalTok{fit }\SpecialCharTok{{-}} \FunctionTok{qnorm}\NormalTok{(}\DecValTok{1}\SpecialCharTok{{-}}\NormalTok{alfa}\SpecialCharTok{/}\DecValTok{2}\NormalTok{)}\SpecialCharTok{*}\NormalTok{pred}\SpecialCharTok{$}\NormalTok{se.fit}
\end{Highlighting}
\end{Shaded}

\begin{verbatim}
##         1 
## 0.8272149
\end{verbatim}

\begin{Shaded}
\begin{Highlighting}[]
\NormalTok{pred}\SpecialCharTok{$}\NormalTok{fit }\SpecialCharTok{+} \FunctionTok{qnorm}\NormalTok{(}\DecValTok{1}\SpecialCharTok{{-}}\NormalTok{alfa}\SpecialCharTok{/}\DecValTok{2}\NormalTok{)}\SpecialCharTok{*}\NormalTok{pred}\SpecialCharTok{$}\NormalTok{se.fit}
\end{Highlighting}
\end{Shaded}

\begin{verbatim}
##        1 
## 1.016706
\end{verbatim}

La mejor opción es predecir el link directamente:

\begin{Shaded}
\begin{Highlighting}[]
\NormalTok{link }\OtherTok{=} \FunctionTok{t}\NormalTok{(xp) }\SpecialCharTok{\%*\%}\NormalTok{ beta\_e}
\NormalTok{(}\AttributeTok{pred =} \FunctionTok{predict}\NormalTok{(m1, }\AttributeTok{newdata =}\NormalTok{ xp, }\AttributeTok{type =} \StringTok{"link"}\NormalTok{, }\AttributeTok{se.fit =}\NormalTok{ T))}
\end{Highlighting}
\end{Shaded}

\begin{verbatim}
## $fit
##        1 
## 2.469289 
## 
## $se.fit
## [1] 0.6718702
## 
## $residual.scale
## [1] 1
\end{verbatim}

Queremos predecir ahora si un Restaurante con Food = 22, Decor = 19,
Service = 24, Price = 55 va a ser clasificado que está en la Guía
Michelín o que no está. Para eso, adoptamos el criterio: si
\(\hat P(Y_p = 1) = \hat \pi_p > 0.5\), entonces \(Y_p = 1\). En este
caso, como \$\hat \pi\_p = \$ 0.9219606, la predicción es que ese
restaurante va a estar incluido en la Guía Michelín. Además, el
intervalo de confianza está muy por encima de 0.5, luego tenemos mucha
confianza en esa decisión.

\hypertarget{variable-respuesta-binomial}{%
\subsection{Variable respuesta
binomial}\label{variable-respuesta-binomial}}

\end{document}
